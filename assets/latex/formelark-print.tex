% Options for packages loaded elsewhere
\PassOptionsToPackage{unicode}{hyperref}
\PassOptionsToPackage{hyphens}{url}
\PassOptionsToPackage{dvipsnames,svgnames,x11names}{xcolor}
%
\documentclass[
  10pt,
  a4paper,
]{article}

\usepackage{amsmath,amssymb}
\usepackage{setspace}
\usepackage{iftex}
\ifPDFTeX
  \usepackage[T1]{fontenc}
  \usepackage[utf8]{inputenc}
  \usepackage{textcomp} % provide euro and other symbols
\else % if luatex or xetex
  \usepackage{unicode-math}
  \defaultfontfeatures{Scale=MatchLowercase}
  \defaultfontfeatures[\rmfamily]{Ligatures=TeX,Scale=1}
\fi
\usepackage{lmodern}
\ifPDFTeX\else  
    % xetex/luatex font selection
    \setmainfont[]{Latin Modern Roman}
\fi
% Use upquote if available, for straight quotes in verbatim environments
\IfFileExists{upquote.sty}{\usepackage{upquote}}{}
\IfFileExists{microtype.sty}{% use microtype if available
  \usepackage[]{microtype}
  \UseMicrotypeSet[protrusion]{basicmath} % disable protrusion for tt fonts
}{}
\makeatletter
\@ifundefined{KOMAClassName}{% if non-KOMA class
  \IfFileExists{parskip.sty}{%
    \usepackage{parskip}
  }{% else
    \setlength{\parindent}{0pt}
    \setlength{\parskip}{6pt plus 2pt minus 1pt}}
}{% if KOMA class
  \KOMAoptions{parskip=half}}
\makeatother
\usepackage{xcolor}
\usepackage[top=1cm, bottom=1.5cm, left=1cm, right=1cm]{geometry}
\setlength{\emergencystretch}{3em} % prevent overfull lines
\setcounter{secnumdepth}{-\maxdimen} % remove section numbering
% Make \paragraph and \subparagraph free-standing
\makeatletter
\ifx\paragraph\undefined\else
  \let\oldparagraph\paragraph
  \renewcommand{\paragraph}{
    \@ifstar
      \xxxParagraphStar
      \xxxParagraphNoStar
  }
  \newcommand{\xxxParagraphStar}[1]{\oldparagraph*{#1}\mbox{}}
  \newcommand{\xxxParagraphNoStar}[1]{\oldparagraph{#1}\mbox{}}
\fi
\ifx\subparagraph\undefined\else
  \let\oldsubparagraph\subparagraph
  \renewcommand{\subparagraph}{
    \@ifstar
      \xxxSubParagraphStar
      \xxxSubParagraphNoStar
  }
  \newcommand{\xxxSubParagraphStar}[1]{\oldsubparagraph*{#1}\mbox{}}
  \newcommand{\xxxSubParagraphNoStar}[1]{\oldsubparagraph{#1}\mbox{}}
\fi
\makeatother
\pagestyle{plain}


\providecommand{\tightlist}{%
  \setlength{\itemsep}{0pt}\setlength{\parskip}{0pt}}\usepackage{longtable,booktabs,array}
\usepackage{calc} % for calculating minipage widths
% Correct order of tables after \paragraph or \subparagraph
\usepackage{etoolbox}
\makeatletter
\patchcmd\longtable{\par}{\if@noskipsec\mbox{}\fi\par}{}{}
\makeatother
% Allow footnotes in longtable head/foot
\IfFileExists{footnotehyper.sty}{\usepackage{footnotehyper}}{\usepackage{footnote}}
\makesavenoteenv{longtable}
\usepackage{graphicx}
\makeatletter
\newsavebox\pandoc@box
\newcommand*\pandocbounded[1]{% scales image to fit in text height/width
  \sbox\pandoc@box{#1}%
  \Gscale@div\@tempa{\textheight}{\dimexpr\ht\pandoc@box+\dp\pandoc@box\relax}%
  \Gscale@div\@tempb{\linewidth}{\wd\pandoc@box}%
  \ifdim\@tempb\p@<\@tempa\p@\let\@tempa\@tempb\fi% select the smaller of both
  \ifdim\@tempa\p@<\p@\scalebox{\@tempa}{\usebox\pandoc@box}%
  \else\usebox{\pandoc@box}%
  \fi%
}
% Set default figure placement to htbp
\def\fps@figure{htbp}
\makeatother

% latex/minimal-header.tex
\usepackage{amsmath, amssymb}
\usepackage{lmodern}
\usepackage{titlesec}

\usepackage{longtable}
\usepackage{tabularx}
\usepackage{booktabs}
\usepackage{array}
\usepackage{caption}
\usepackage{ltablex}  % longtable + tabularx
\keepXColumns



% Remove spacing around title
\makeatletter
\renewcommand{\maketitle}{%
    \begin{center}
    {\Large\bfseries \@title \par}
    \vspace{0.5em}
    \end{center}
}
\makeatother

% Remove section spacing and numbering
\titleformat{\section}[block]{\bfseries\large}{}{0pt}{}
\titlespacing*{\section}{0pt}{1ex plus 0.2ex minus .2ex}{0.5ex plus .1ex}


% Needed for rendering the formelark-print.qmd to PDF
\newcommand{\blockheader}[1]{\noalign{\vskip 1.2ex}%
\multicolumn{2}{@{}l@{}}{\normalsize\bfseries\large #1} \\
\addlinespace}

\renewcommand{\arraystretch}{1.3}
\small
\makeatletter
\@ifpackageloaded{caption}{}{\usepackage{caption}}
\AtBeginDocument{%
\ifdefined\contentsname
  \renewcommand*\contentsname{Table of contents}
\else
  \newcommand\contentsname{Table of contents}
\fi
\ifdefined\listfigurename
  \renewcommand*\listfigurename{List of Figures}
\else
  \newcommand\listfigurename{List of Figures}
\fi
\ifdefined\listtablename
  \renewcommand*\listtablename{List of Tables}
\else
  \newcommand\listtablename{List of Tables}
\fi
\ifdefined\figurename
  \renewcommand*\figurename{Figure}
\else
  \newcommand\figurename{Figure}
\fi
\ifdefined\tablename
  \renewcommand*\tablename{Table}
\else
  \newcommand\tablename{Table}
\fi
}
\@ifpackageloaded{float}{}{\usepackage{float}}
\floatstyle{ruled}
\@ifundefined{c@chapter}{\newfloat{codelisting}{h}{lop}}{\newfloat{codelisting}{h}{lop}[chapter]}
\floatname{codelisting}{Listing}
\newcommand*\listoflistings{\listof{codelisting}{List of Listings}}
\makeatother
\makeatletter
\makeatother
\makeatletter
\@ifpackageloaded{caption}{}{\usepackage{caption}}
\@ifpackageloaded{subcaption}{}{\usepackage{subcaption}}
\makeatother

\usepackage{bookmark}

\IfFileExists{xurl.sty}{\usepackage{xurl}}{} % add URL line breaks if available
\urlstyle{same} % disable monospaced font for URLs
\hypersetup{
  pdftitle={Vedlegg til eksamen i BED3 Investering og finans},
  colorlinks=true,
  linkcolor={blue},
  filecolor={Maroon},
  citecolor={Blue},
  urlcolor={Blue},
  pdfcreator={LaTeX via pandoc}}


\title{Vedlegg til eksamen i BED3 Investering og finans}
\author{}
\date{}

\begin{document}
\maketitle


\setstretch{1}
\renewcommand{\arraystretch}{2.0}
\normalsize
\begin{tabularx}{\textwidth}{@{}lX@{}}
\multicolumn{2}{@{}l@{}}{\textbf{Tidens verdi}} \\ \addlinespace
Fremtidsverdi av ett beløp: & $ FV = CF_0 \cdot (1 + k)^T $ \\
Nåverdi av ett beløp: & $ PV = \frac{CF_T}{(1 + k)^T} $ \\
Annuitet (endelig horisont): & $ ACF = CF \cdot \frac{(1 + k)^T -1}{k \cdot (1 + k)^T} = CF \cdot \frac{1- \left(\frac{1}{1 +k}\right)^T}{k} = CF \cdot A_{k,T} $ \\
Annuitet (uendelig horisont): & $ ACF =  \frac{CF}{k} $ \\
Fordeling av et beløp til annuitet: & $ CF_0 \cdot \frac{k\cdot (1 + k)^T}{(1 + k)^T - 1} = CF_{0} \cdot \frac{k}{1-\left(\frac{1}{1 + k}\right)^T} = CF_{0} \cdot A_{k,T}^{-1} $ \\
Vekstrekke (endelig horisont): & $ PV = CF_1 \cdot \frac{1 - \left( \frac{1 + g}{1 + k} \right)^T}{k - g}, \quad \text{for } k \neq g $ \\
Vekstrekke (uendelig horisont): & $ PV = \frac{CF_1}{k - g}, \quad \text{for } g < k $ \\
\multicolumn{2}{@{}l@{}}{\textbf{Nåverdimetoder}} \\ \addlinespace
Netto nåverdi (NPV): & $ NPV = - I_0 + \sum_{t = 1}^{T} \frac{CF_t}{(1 + k)^t} $ \\
Internrente (IRR): & $ 0 = \sum_{t = 1}^{T} \frac{CF_t}{(1 + y)^t} - I_0 $ \\
Annuitetsmetoden: & $ CF - I_{0} \cdot A_{k,T}^{-1} = NPV \cdot A_{k,T}^{-1} $ \\
Nåverdiindeks (PVI): & $ PVI = \frac{NPV}{I_0} $ \\
Payback uten renter: & $ PB = \frac{I_0}{CF} $ \\
Payback med renter (annuitet): & $ 0 = -I_0 + CF \cdot A_{k, PB} $ \\
\multicolumn{2}{@{}l@{}}{\textbf{Skatt og avskrivninger}} \\ \addlinespace
Effektiv skattesats: & $ s_{\text{Eff}} = \frac{y - y^s}{y} $ \\
Saldoavskrivningsbeløp: & $ AV_t = I_0 \cdot (1 - a)^{t - 1} \cdot a $ \\
Bokført restverdi: & $ B_t = I_0 \cdot (1 - a)^t $ \\
Skattefordel av avskrivninger: & $ PV_s = s \cdot \frac{I_0 \cdot a}{k^s + a} = s \cdot \frac{AV_1}{k^s + a} $ \\
Utrangering (nåverdi av sluttverdi med skatt): & $ PV_T = \frac{S_T}{(1 + k^s)^T} - s \cdot \frac{(S_T - B_T) \cdot a}{(1 + k^s)^T \cdot (k^s + a)} $ \\
\multicolumn{2}{@{}l@{}}{\textbf{Renteregning}} \\ \addlinespace
Fra nominell til reell rente: & $ k_R = \frac{k_N - i}{1 + i} $ \\
Fra reell til nominell rente: & $ k_N = k_R \cdot (1 + i) + i $ \\
Fra nominell lånerente før skatt til reell lånerente etter skatt: & $ y_R^s = \frac{y_N \cdot (1-s) -i}{1+i} $ \\
Fra delperioderente til effektiv rente: & $ p = (1 + q)^m - 1 $ \\
Fra effektiv rente til delperioderente: & $ q = (1 + p)^{1/m} - 1 $ \\
Fra forskuddsrente til etterskuddsrente: & $ q_e = \frac{q_f}{1 - q_f} $ \\
Fra etterskuddsrente til forskuddsrente: & $ q_f = \frac{q_e}{1 + q_e} $ \\
\multicolumn{2}{@{}l@{}}{\textbf{Porteføljeteori og risiko}} \\ \addlinespace
Forventet avkastning, enkeltaktivum: & $ E(r_i) = \sum_{j = 1}^{n} p_j \cdot r_{ij} $ \\
Varians, enkeltaktivum: & $ \sigma_i^2 = \sum_{j = 1}^{n} p_j \cdot \left[ r_{ij} - E(r_i) \right]^2 $ \\
Standardavvik, enkeltaktivum: & $ \sigma_i = \sqrt{\sigma_i^2} $ \\
Forventet avkastning, portefølje to aktiva: & $ E(r_p) = w_1 \cdot E(r_1) + (1 - w_1) \cdot E(r_2) $ \\
Varians, portefølje med to aktiva: & $ \sigma_p^2 = w_1^2 \cdot \sigma_1^2 + (1 - w_1)^2 \cdot \sigma_2^2 + 2 \cdot w_1 \cdot (1 - w_1) \cdot \sigma_{12} $ \\
Kovarians mellom to aktiva (korrelasjonsformel): & $ \sigma_{12} = \rho_{12} \cdot \sigma_1 \cdot \sigma_2 $ \\
Kovarians mellom to aktiva (diskret fordeling): & $ \sigma_{12} = \sum_{j = 1}^{n} p_j \cdot \left[ r_{1j} - E(r_1) \right] \cdot \left[ r_{2j} - E(r_2) \right] $ \\
Minimumsvariansportefølje – andel i aktivum 1: & $ w_1^* = \frac{\sigma_2^2 - \sigma_{12}}{\sigma_1^2 + \sigma_2^2 - 2\sigma_{12}} $ \\
Forventet avkastning, portefølje med N aktiva: & $ E(r_p) = \sum_{i = 1}^{N} w_i \cdot E(r_i) $ \\
Varians, portefølje med N aktiva: & $ \sigma_p^2 = \sum_{i = 1}^{N} \sum_{j = 1}^{N} w_i \cdot w_j \cdot \sigma_{ij} $ \\
Kapitalmarkedslinjen (CML): & $ E(r_P) = r_f + \frac{E(r_M) - r_f}{\sigma_M} \cdot \sigma_P $ \\
\multicolumn{2}{@{}l@{}}{\textbf{Kapitalverdimodellen}} \\ \addlinespace
CAPM (grunnform): & $ E(r_i) = r_f + \beta_i \cdot \left[ E(r_M) - r_f \right] $ \\
CAPM (skattejustert): & $ E(r_i) = r_f \cdot (1-s) + \beta_i \cdot \left[ E(r_M) - r_f \cdot (1 - s) \right] $ \\
Beta: & $ \beta_i = \frac{\sigma_{iM}}{\sigma_M^2} = \rho_{iM} \cdot \frac{\sigma_i}{\sigma_M} $ \\
Alfa: & $ \alpha_i = E(r_i) - \left[ r_f + \beta_i \cdot \left( E(r_M) - r_f \right) \right] $ \\
Variansdekomponering: & $ \sigma_i^2 = \beta_i^2 \cdot \sigma_M^2 + \sigma_{\varepsilon i}^2 $ \\
\multicolumn{2}{@{}l@{}}{\textbf{Verdsettelse}} \\ \addlinespace
Dividendemodellen (generell): & $ P_0 = \sum_{t = 1}^{T} \frac{DIV_t}{(1 + k_E)^t} + \frac{P_T}{(1+k_E)^T} $ \\
Dividendemodellen (nullvekst): & $ P_0 = \frac{DIV_1}{k_E} $ \\
Dividendemodellen (evigvarende konstant vekst): & $ P_0 = \frac{DIV_1}{k_E - g}, \quad \text{for } g < k_E $ \\
Vekst (intern generert): & $ g = R_E \cdot b $ \\
Pris/Fortjeneste-modellen (P/E): & $ P/E \triangleq \frac{P_0}{EPS_1} = \frac{d}{k_E - g} $ \\
Vekstmuligheter (PVGO): & $ P_0 = \frac{EPS_1}{k_E} + PVGO $ \\
\multicolumn{2}{@{}l@{}}{\textbf{Kapitalstruktur og vektstang}} \\ \addlinespace
Vektstangsformel – rentabilitet EK: & $ R_E = R_T + \frac{G}{E} \cdot (R_T - k_G) $ \\
Vektstangsformel – risiko på egenkapital (standardavvik): & $ \sigma_{R_E} = \sigma_{R_T} \cdot \left( 1 + \frac{G}{E} \right) $ \\
Vektstangsformel – før skatt: & $ k_T = \frac{E}{E + G} \cdot k_E + \frac{G}{E + G} \cdot k_G $ \\
Vektstangsformel – etter skatt: & $ k_T^s = \frac{E}{E + G} \cdot k_E + \frac{G}{E + G} \cdot k_G \cdot (1 - s) $ \\
Vektstangsformel – krav egenkapital: & $ k_E = k_T + \frac{G}{E} \cdot (k_T - k_G) $ \\
Beta for totalkapital: & $ \beta_T = \frac{E}{E + G} \cdot \beta_E + \frac{G}{E + G} \cdot \beta_G $ \\
Beta for egenkapital (dersom $\beta_G = 0$): & $ \beta_E = \beta_T \cdot \left( 1 + \frac{G}{E} \right) $ \\
Årlig skattefordel ved gjeld pr. krone: & $ \text{Skattefordel} =  (1 - s_K) - (1 - s_{B}) \cdot (1 - s_{Ed}) $ \\
Årlig skattefordel ved dividende pr. krone: & $ \text{Skattefordel} = (1 - s_{B}) \cdot (s_{Eg} - s_{Ed}) $ \\
\multicolumn{2}{@{}l@{}}{\textbf{Obligasjoner og durasjon}} \\ \addlinespace
Obligasjonspris (standardformel): & $ P_0 = \sum_{t = 1}^{T} \frac{c}{(1 + y)^t} + \frac{\text{Pålydende}}{(1 + y)^T} $ \\
Obligasjonspris (med konstant kupong og helt antall år til forfall): & $ P_0 = \frac{c}{y} \cdot \left[1 - \frac{1}{(1+y)^T}\right] + \frac{\text{Pålydende}}{(1+y)^T} = c \cdot A_{y,T} + \frac{\text{Pålydende}}{(1+y)^T} $ \\
Sertifikatpris: & $ P_0 = \frac{P_1}{1 + r \cdot \frac{d}{365}} \quad \quad P_0 = \frac{P_1}{(1 + y)^{\frac{d}{365}}} $ \\
Terminrente mellom t−1 og t: & $ f_{t-1,t} = \frac{(1 + r_t)^t}{(1 + r_{t-1})^{t-1}} - 1 $ \\
Durasjon: & $ D = \frac{1}{P_0} \cdot \sum_{t = 1}^{T} t \cdot \frac{CF_t}{(1 + y)^t} $ \\
Justert durasjon (modifisert): & $ D^* = \frac{-D}{1 + y} $ \\
Durasjonsbasert prisendring: & $ \frac{\Delta P}{P} =  \frac{-D}{1 + y} \cdot \Delta y = D^* \cdot \Delta y $ \\
\multicolumn{2}{@{}l@{}}{\textbf{Opsjoner og derivater}} \\ \addlinespace
Black-Scholes (kjøpsopsjon): & $ C = S_0 \cdot N(d_1) - K \cdot e^{-r_f \cdot T} \cdot N(d_2) $ \\
Parametere i Black-Scholes: & $ d_1 = \frac{\ln \left( \frac{S_0}{K} \right) + r_f \cdot T}{\sigma \sqrt{T}} + \frac{1}{2} \cdot \sigma \cdot \sqrt{T}, \quad_2 = d_1 - \sigma \sqrt{T} $ \\
Put-call paritet: & $ C_0 - P_0 = S_0 - PV(K) $ \\
Lagringskostnadshypotesen (generell form): & $ F_T = S_0 \cdot (1 + c \cdot T) $ \\
Lagringskostnadshypotesen (aksjer med utbytte): & $ F_T = S_0 \cdot \left[1 + (r_f - \text{div}) \cdot T\right] $ \\
Forventningshypotesen (terminpriser): & $ F_T = E(S_T) $ \\
\multicolumn{2}{@{}l@{}}{\textbf{Internasjonal finans}} \\ \addlinespace
Relativ kjøpekraftsparitet (PPP): & $ \frac{E(s_{\text{UTL/NOK}})}{s_{\text{UTL/NOK}}} = \frac{1 + i_{\text{NOK}}}{1 + i_{\text{UTL}}} $ \\
Dekket renteparitet (CIP): & $ \frac{1 + r_{\text{NOK}}}{1 + r_{\text{UTL}}} = \frac{f_{\text{UTL/NOK}}}{s_{\text{UTL/NOK}}} $ \\
Udekket renteparitet (UIP): & $ \frac{1 + r_{\text{NOK}}}{1 + r_{\text{UTL}}} = \frac{E(s_{\text{UTL/NOK}})}{s_{\text{UTL/NOK}}} $ \\
Fishereffekten (to valutaer): & $ \frac{1 + r_{\text{NOK}}}{1 + r_{\text{UTL}}} = \frac{E(1 + i_{\text{NOK}})}{E(1 + i_{\text{UTL}})} $ \\
Forventningshypotesen (valuta): & $ \frac{f_{\text{UTL/NOK}}}{s_{\text{UTL/NOK}}} = \frac{E(s_{\text{UTL/NOK}})}{s_{\text{UTL/NOK}}} \Rightarrow f_{\text{UTL/NOK}} = E(s_{\text{UTL/NOK}}) $ \\
Avkastningskrav ved utenlandske investeringer: & $ 1 + k_{\text{NOK}} = (1 + k_{\text{UTL}}) \cdot \frac{1 + r_{\text{NOK}}}{1 + r_{\text{UTL}}} $ \\
Lånekostnad ved utenlandske lån: & $ \frac{s_{\text{UTL/NOK,1}}}{s_{\text{UTL/NOK,0}}} \cdot (1 + r_{\text{UTL}}) - 1 $ \\
\end{tabularx}




\end{document}
